\documentclass[12pt,a4paper]{scrartcl}		% KOMA-Klassen benutzen!

\usepackage[utf8]{inputenc}			% Zeichensatzkodierung


\newcommand\svthema{INF264 Project 3}
\newcommand\svperson{Sophie Blum and Benjamin Friedl}
\newcommand\svdatum{30.10.2020}
\newcommand\lvname{Digit recognizer - Summary}
\begin{document}

\title{ \svthema}
\author{\textsc{\lvname}}
\date{ \small \textsl{\svperson} --- \svdatum }
\maketitle

In this project, we computed different machine learning solutions to classify handwritten 
digits based on a given dataset. This included pre-processing the data as well as training 
different kinds of models with different values for chosen hyper-parameters. Whilst we also 
tested a K-Nearest-Neighbour-Classifier and a Support-Vector-Machine-Classifier, the best 
results were produced by a Convolutional Neural Network. With an accuracy of 98.09\% 
on an unseen test-dataset, this turned out to be a viable approach for this task. The 
expected perfomance in real life is expected to be smaller, but in a similar range, as the 
perfomance on the test data is usually an overestimation of the model as well.

\end{document}